Figure {block_diagram} [book] presents the schematic diagram of a general WPT system. The operating principle can be classified into \textit{maximum power transfer} and \textit{maximum power efficiency transfer} [A Critical]. This article is based on the efficiency criterion to reach a compromise between the harvested power and the physical limitation. As suggested in [book], the overall power transmit efficiency $e$ can be decomposed as

\begin{equation}\label{eqn:power_utilization_efficiency}
  e = \frac{{{P_{{\text{dc}},{\text{ST}}}}}}{{P_{{\text{dc}}}^t}} = \underbrace {\frac{{P_{{\text{rf}}}^t}}{{P_{{\text{dc}}}^t}}}_{{e_1}} \cdot \underbrace {\frac{{P_{{\text{rf}}}^r}}{{P_{{\text{rf}}}^t}}}_{{e_2}} \cdot \underbrace {\frac{{P_{{\text{dc}}}^r}}{{P_{{\text{rf}}}^r}}}_{{e_3}} \cdot \underbrace {\frac{{{P_{{\text{dc}},{\text{ST}}}}}}{{P_{{\text{dc}}}^r}}}_{{e_4}}
\end{equation}

where the power transmitter determines the DC-to-RF efficiency ${e_1}$, the channel influences the RF-to-RF efficiency ${e_2}$, the rectenna decides the RF-to-DC efficiency ${e_3}$, and the power management unit (PMU) deals with the DC-to-DC efficiency ${e_4}$. Most existing solutions assume no dependency in between and focus on maximizing each term individually [see ref] then concatenate them together. Specifically, ${e_1}$ ${e_3}$ and ${e_4}$ are often neglected in waveform design and resource allocation. Interestingly, it has been proved by [Communications and][Waveform design][Practical nonlinear] that these efficiencies are coupled with each other, and the problem requires a joint optimization. Indeed, the RF-to-DC efficiency ${e_3}$ is observed to be a nonlinear function of the rectifier input power ${P_{{\text{rf}}}^r}$ [Towards the][Maximum achievable][Power-optimized], which also depends on signal waveform [Optimum waveform][Boosting the]. This article employs a tractable nonlinear harvester model proposed in [Waveform design] that was proved beneficial to the harvested current. 