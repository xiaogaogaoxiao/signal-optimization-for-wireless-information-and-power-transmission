For the MISO case, the optimal phase from rate and energy perspectives can be derived from the mutual information in \ref{eqn:mutual_information}, the target function in \ref{eqn:target_function_truncated}, and the waveform expressions in \ref{eqn:power_waveform_second_order} -- \ref{eqn:waveform_end}. The solutions correspond to the matched filters w.r.t. the phases of the channel

\begin{equation}\label{eqn:optimal_phases}
  \phi _{I,n,m}^ \star  = \phi _{P,n,m}^ \star  =  - {{\bar \psi }_{n,m}}
\end{equation}

Such phase decisions can guarantee all the cosine terms in \ref{eqn:power_waveform_second_order} -- \ref{eqn:waveform_end} are maximized by setting the arguments to 0. ${\mathbf{\Phi }}_I^ \star $ and ${\mathbf{\Phi }}_P^ \star $ can be constructed by collecting $\phi _{I,n,m}^ \star $ and $\phi _{P,n,m}^ \star $ to the $(n,m)$ entries respectively. On top of that, the target function ${z_{DC}}\left( {{{\mathbf{S}}_P},{{\mathbf{S}}_I},{\mathbf{\Phi }}_P^ \star ,{\mathbf{\Phi }}_I^ \star ,\rho } \right)$ can be further written as

\begin{align}\label{eqn:target_posynomial}
  {z_{DC}} &= \frac{{{k_2}\rho }}{2}{R_{{\text{ant}}}}\sum\limits_{n = 0}^{N - 1} {\sum\limits_{{m_0},{m_1}} {\left[ {\prod\limits_{j = 0}^1 {{s_{P,n,{m_j}}}} {A_{n,{m_j}}}} \right]} } \nonumber \\
   &\quad + \frac{{3{k_4}{\rho ^2}}}{8}R_{{\text{ant}}}^2\sum\limits_{\substack{ {n_0},{n_1},{n_2},{n_3} \\ {n_0} + {n_1} = {n_2} + {n_3} } }  {\sum\limits_{{m_0},{m_1},{m_2},{m_3}} {\left[ {\prod\limits_{j = 0}^3 {{s_{P,{n_j},{m_j}}}{A_{{n_j},{m_j}}}} } \right]} } \nonumber \\
   &\quad + \frac{{{k_2}\rho }}{2}{R_{{\text{ant}}}}\sum\limits_{n = 0}^{N - 1} {\sum\limits_{{m_0},{m_1}} {\left[ {\prod\limits_{j = 0}^1 {{s_{I,n,{m_j}}}} {A_{n,{m_j}}}} \right]} } \nonumber \\
   &\quad + \frac{{3{k_4}{\rho ^2}}}{4}R_{{\text{ant}}}^2\sum\limits_{{n_0},{n_1}} {\sum\limits_{{m_0},{m_1},{m_2},{m_3}} {\left[ {\prod\limits_{j = 0,2} {{s_{I,{n_0},{m_j}}}{A_{{n_0},{m_j}}}} } \right]\left[ {\prod\limits_{j = 1,3} {{s_{I,{n_1},{m_j}}}{A_{{n_1},{m_j}}}} } \right]} } \nonumber \\
   &\quad + \frac{{3{k_4}{\rho ^2}}}{2}R_{\text{ant}}^2\left[ {\sum\limits_{n = 0}^{N - 1} {\sum\limits_{{m_0},{m_1}} {\left[ {\prod\limits_{j = 0}^1 {{s_{P,n,{m_j}}}} {A_{n,{m_j}}}} \right]} } } \right]\left[ {\sum\limits_{n = 0}^{N - 1} {\sum\limits_{{m_0},{m_1}} {\left[ {\prod\limits_{j = 0}^1 {{s_{I,n,{m_j}}}} {A_{n,{m_j}}}} \right]} } } \right]
\end{align}

Similarly, the mutual information is given by

\begin{equation}\label{eqn:mutual_information_posynomial}
  I\left( {{{\mathbf{S}}_I},{\mathbf{\Phi }}_I^ \star ,\rho } \right) = {\log _2}\left( {\prod\limits_{n = 0}^{N - 1} {\left( {1 + \frac{{(1 - \rho )}}{{\sigma _n^2}}{C_n}} \right)} } \right)
\end{equation}

where ${C_n} = \sum\nolimits_{{m_0},{m_1}} {\prod\nolimits_{j = 0}^1 {{s_{I,n,{m_j}}}{A_{n,{m_j}}}} } $.

Note both the target function and mutual information are posynomials \cite{Boyd2007}. Therefore, one possible approach to characterize the rate-energy region is to transform the optimization into an energy maximization problem with average transmit power budget $P$ and rate constraint ${\bar R}$

\begin{eqnarray}
  {\mathop {\max }\limits_{{{\mathbf{S}}_P},{{\mathbf{S}}_I},\rho } }&{{z_{DC}}\left( {{{\mathbf{S}}_P},{{\mathbf{S}}_I},{\mathbf{\Phi }}_P^ \star ,{\mathbf{\Phi }}_I^ \star ,\rho } \right)} \label{eqn:original_target}\\
  {{\text{ subject to }}}&{\frac{1}{2}\left[ {\left\| {{{\mathbf{S}}_I}} \right\|_F^2 + \left\| {{{\mathbf{S}}_P}} \right\|_F^2} \right] \leqslant P,} \label{eqn:original_power_constraint} \\
  {}&{I\left( {{{\mathbf{S}}_I},{\mathbf{\Phi }}_I^ \star ,\rho } \right) \geqslant \bar R} \label{eqn:original_rate_constraint}
\end{eqnarray}

Although the problem is not a standard Geometric Programming (GP), we can transform it to a Reversed GP by introducing an auxiliary variable ${t_0}$ \cite{Chiang2005}

\begin{eqnarray}
  {\mathop {\min }\limits_{{{\mathbf{S}}_P},{{\mathbf{S}}_I},\rho ,{t_0}} }&{1/{t_0}} \label{eqn:transformed_target} \\
  {{\text{ subject to }}}&{\frac{1}{2}\left[ {\left\| {{{\mathbf{S}}_I}} \right\|_F^2 + \left\| {{{\mathbf{S}}_P}} \right\|_F^2} \right] \leqslant P} \label{eqn:transformed_power_constraint} \\
  {}&{{t_0}/{z_{DC}}\left( {{{\mathbf{S}}_P},{{\mathbf{S}}_I},{\mathbf{\Phi }}_P^ \star ,{\mathbf{\Phi }}_I^ \star ,\rho } \right) \leqslant 1} \label{eqn:transformed_current_constraint} \\
  {}&{{2^{\bar R}}/\left[ {\prod\limits_{n = 0}^{N - 1} {\left( {1 + \frac{{(1 - \rho )}}{{\sigma _n^2}}{C_n}} \right)} } \right] \leqslant 1} \label{eqn:transformed_rate_constraint}
\end{eqnarray}

The new problem so far is not a standard GP as $1/{z_{DC}}\left( {{{\mathbf{S}}_P},{{\mathbf{S}}_I},{\mathbf{\Phi }}_P^ \star ,{\mathbf{\Phi }}_I^ \star ,\rho } \right)$ and $1/\left[ {\prod\nolimits_{n = 0}^{N - 1} {\left( {1 + \frac{{(1 - \rho )}}{{\sigma _n^2}}{C_n}} \right)} } \right]$ are not posynomials. To solve this, \cite{Clerckx2018} suggested a conservative approach to approximate the terms with posynomials in the denominator by new posynomials, based on the Arithmetic Mean-Geometric Mean (AM-GM) inequality.

Consider constraint \ref{eqn:transformed_current_constraint} first. The posynomial at the denominator can be decomposed as sum of monomials

\begin{equation}\label{eqn:transformed_current_posynomial_decomposition}
  {z_{DC}}\left( {{{\mathbf{S}}_P},{{\mathbf{S}}_I},{\mathbf{\Phi }}_P^ \star ,{\mathbf{\Phi }}_I^ \star ,\rho } \right) = \sum\limits_{k = 1}^K {{g_k}\left( {{{\mathbf{S}}_P},{{\mathbf{S}}_I},{\mathbf{\Phi }}_P^ \star ,{\mathbf{\Phi }}_I^ \star ,\rho } \right)} 
\end{equation}

Since monomial $\left\{ {{g_k}} \right\}$ is nonnegative for all $k$, the AM-GM inequality suggests a posynomial upper bound for the previous non-posynomial term

\begin{equation}\label{eqn:transformed_current_am_gm}
  \frac{1}{{\sum\limits_{k = 1}^K {{g_k}\left( {{{\mathbf{S}}_P},{{\mathbf{S}}_I},{\mathbf{\Phi }}_P^ \star ,{\mathbf{\Phi }}_I^ \star ,\rho } \right)} }} \leqslant \prod\limits_{k = 1}^K {{{\left( {\frac{{{g_k}\left( {{{\mathbf{S}}_P},{{\mathbf{S}}_I},{\mathbf{\Phi }}_P^ \star ,{\mathbf{\Phi }}_I^ \star ,\rho } \right)}}{{{\gamma _k}}}} \right)}^{ - {\gamma _k}}}} 
\end{equation}

The nonnegative coefficients $\left\{ {{\gamma _k}} \right\}$ are chosen to satisfy $\sum\nolimits_{k = 1}^K {{\gamma _k}}  = 1$. Similarly, define $\bar \rho  = 1 - \rho $ and let $\left\{ {{g_{nk}}\left( {{{\mathbf{S}}_I},\bar \rho } \right)} \right\}$ be the monomials of the posynomial $1 + \frac{{\bar \rho }}{{\sigma _n^2}}{C_n}$

\begin{equation}\label{eqn:transformed_rate_posynomial_decomposition}
  1 + \frac{{\bar \rho }}{{\sigma _n^2}}{C_n} = \sum\limits_{k = 1}^{{K_n}} {{g_{nk}}} \left( {{{\mathbf{S}}_I},\bar \rho } \right)
\end{equation}

Apply the AM-GM inequality to \ref{eqn:transformed_rate_posynomial_decomposition}, we have

\begin{equation}\label{eqn:transformed_rate_am_gm}
  \frac{1}{{1 + \frac{{\bar \rho }}{{\sigma _n^2}}{C_n}}} \leqslant \prod\limits_{k = 1}^{{K_n}} {{{\left( {\frac{{{g_{nk}}\left( {{{\mathbf{S}}_I},\bar \rho } \right)}}{{{\gamma _{nk}}}}} \right)}^{ - {\gamma _{nk}}}}} 
\end{equation}

with ${\gamma _{nk}} \geqslant 0$ and $\sum\nolimits_{k = 1}^{{K_n}} {{\gamma _{nk}}}  = 1$. In this way, we transformed the problem into a standard GP

\begin{eqnarray}
  {\mathop {\min }\limits_{{{\mathbf{S}}_P},{{\mathbf{S}}_I},\rho ,\bar \rho ,{t_0}} }&{1/{t_0}} \label{eqn:general_target} \\
  {{\text{ subject to }}}&{\frac{1}{2}\left[ {\left\| {{{\mathbf{S}}_I}} \right\|_F^2 + \left\| {{{\mathbf{S}}_P}} \right\|_F^2} \right] \leqslant P} \label{eqn:general_power_constraint} \\
  {}&{{t_0}\prod\limits_{k = 1}^K {{{\left( {\frac{{{g_k}\left( {{{\mathbf{S}}_P},{{\mathbf{S}}_I},{\mathbf{\Phi }}_P^ \star ,{\mathbf{\Phi }}_I^ \star ,\rho } \right)}}{{{\gamma _k}}}} \right)}^{ - {\gamma _k}}}}  \leqslant 1} \label{eqn:general_current_constraint} \\
  {}&{2^{\bar R}}\prod\limits_{n = 0}^{N - 1} {\prod\limits_{k = 1}^{{K_n}} {{{\left( {\frac{{{g_{nk}}\left( {{{\mathbf{S}}_I},\bar \rho } \right)}}{{{\gamma _{nk}}}}} \right)}^{ - {\gamma _{nk}}}}} }  \leqslant 1 \label{eqn:general_rate_constraint} \\
  {}&{\rho  + \bar \rho  \leqslant 1} \label{eqn:general_ratio_constraint} 
\end{eqnarray}

It is worth noting that the tightness of the AM-GM inequality depends on the choice of $\left\{ {{\gamma _k},{\gamma _{nk}}} \right\}$. In this paper, we employ the iterative method proposed in \cite{Clerckx2018} that updates the coefficient sets at iteration $i$ with the previous solution ${{\mathbf{S}}_P^{(i - 1)},{\mathbf{S}}_I^{(i - 1)},{\rho ^{(i - 1)}}}$

\begin{eqnarray}
  {{\gamma _k} = \frac{{{g_k}\left( {{\mathbf{S}}_P^{(i - 1)},{\mathbf{S}}_I^{(i - 1)},{\rho ^{(i - 1)}}} \right)}}{{{z_{DC}}\left( {{\mathbf{S}}_P^{(i - 1)},{\mathbf{S}}_I^{(i - 1)},{\rho ^{(i - 1)}}} \right)}},}&{k = 1, \ldots ,K} \\
  {{\gamma _{nk}} = \frac{{{g_{nk}}\left( {{\mathbf{S}}_I^{(i - 1)},{{\bar \rho }^{(i - 1)}}} \right)}}{{1 + \frac{{{{\bar \rho }^{(i - 1)}}}}{{\sigma _n^2}}{C_n}\left( {{\mathbf{S}}_I^{(i - 1)}} \right)}},}&\begin{gathered}
  n = 0, \ldots ,N - 1 \hfill \\
  k = 1, \ldots ,{K_n} \hfill \\
\end{gathered}
\end{eqnarray}

Once $\left\{ {{\gamma _k},{\gamma _{nk}}} \right\}$ are obtained, we solve problem \ref{eqn:general_target} -- \ref{eqn:general_ratio_constraint} to obtain ${\mathbf{S}}_P^{(i)},{\mathbf{S}}_I^{(i)},{\rho ^{(i)}}$. The iterations are repeated until convergence. Algorithm 1 summarizes all the procedures involved in the optimization. The successive approximation approach is also known as inner approximation method \cite{Marks1978}, which cannot guarantee a global optimal solution but the converged point satisfies the KKT conditions.  