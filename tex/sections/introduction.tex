Wireless networks have been restricted by the drawbacks of batteries as environmental hazards, limited operation time, and high cost in recharging or replacement. As a promising solution, energy harvesting (EH) from the ambient environment potentially provides perpetual power to the energy-constrained devices. Renewable energy sources as solar, wind and water have been widely employed for power supply, while the energy carried by radio-frequency (RF) waves is not as significant and more suitable for low-power applications. 


Wireless power transmission (WPT) that enables harvesters to collect energy from the electromagnetic (EM) waves has received recent attentions in both academia and industry \cite{R.Varshney2008,Grover2010,Zhang2013,Hui2014,Krikidis2014,Valenta2014,Boshkovska2015,Ding2015,Costanzo2016,Clerckx2018a}.



 It has become a prominent solution to power billions of energy-constrained devices while keeping them connected.





Compared with other renewable energy sources as solar, water and wind, radio-frequency (RF) signals carry both information and energy simultaneously. 