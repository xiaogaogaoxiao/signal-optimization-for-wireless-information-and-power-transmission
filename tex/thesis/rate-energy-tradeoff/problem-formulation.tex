For the MISO case, the optimal phases to maximize the target function \eqref{eqn:target_function_truncated} and the mutual information \eqref{eqn:mutual_information} correspond to the phases of the matched filters

\begin{equation}\label{eqn:optimal_phases}
  \phi _{P,n,m}^ \star  = \phi _{I,n,m}^ \star  =  - {{\bar \psi }_{n,m}}
\end{equation}

Such a decision can maximize all cosine terms in \eqref{eqn:power_waveform_second_order} -- \eqref{eqn:waveform_end} by setting the arguments to 0. ${\mathbf{\Phi }}_P^ \star $ and ${\mathbf{\Phi }}_I^ \star $ can be constructed by collecting $\phi _{P,n,m}^ \star $ and $\phi _{I,n,m}^ \star $ to the $(n,m)$ entries respectively. On top of it, the target function ${z_{DC}}\left( {{{\mathbf{S}}_P},{{\mathbf{S}}_I},{\mathbf{\Phi }}_P^ \star ,{\mathbf{\Phi }}_I^ \star ,\rho } \right)$ can be further written as

\begin{align}\label{eqn:target_posynomial}
  {z_{DC}} &= \frac{{{k_2}\rho }}{2}{R_{{\text{ant}}}}\sum\limits_{n = 0}^{N - 1} {\sum\limits_{{m_0},{m_1}} {\left[ {\prod\limits_{j = 0}^1 {{s_{P,n,{m_j}}}} {A_{n,{m_j}}}} \right]} } \nonumber \\
   &\quad + \frac{{3{k_4}{\rho ^2}}}{8}R_{{\text{ant}}}^2\sum\limits_{\substack{ {n_0},{n_1},{n_2},{n_3} \\ {n_0} + {n_1} = {n_2} + {n_3} } }  {\sum\limits_{{m_0},{m_1},{m_2},{m_3}} {\left[ {\prod\limits_{j = 0}^3 {{s_{P,{n_j},{m_j}}}{A_{{n_j},{m_j}}}} } \right]} } \nonumber \\
   &\quad + \frac{{{k_2}\rho }}{2}{R_{{\text{ant}}}}\sum\limits_{n = 0}^{N - 1} {\sum\limits_{{m_0},{m_1}} {\left[ {\prod\limits_{j = 0}^1 {{s_{I,n,{m_j}}}} {A_{n,{m_j}}}} \right]} } \nonumber \\
   &\quad + \frac{{3{k_4}{\rho ^2}}}{4}R_{{\text{ant}}}^2\sum\limits_{{n_0},{n_1}} {\sum\limits_{{m_0},{m_1},{m_2},{m_3}} {\left[ {\prod\limits_{j = 0,2} {{s_{I,{n_0},{m_j}}}{A_{{n_0},{m_j}}}} } \right]\left[ {\prod\limits_{j = 1,3} {{s_{I,{n_1},{m_j}}}{A_{{n_1},{m_j}}}} } \right]} } \nonumber \\
   &\quad + \frac{{3{k_4}{\rho ^2}}}{2}R_{\text{ant}}^2\left[ {\sum\limits_{n = 0}^{N - 1} {\sum\limits_{{m_0},{m_1}} {\left[ {\prod\limits_{j = 0}^1 {{s_{P,n,{m_j}}}} {A_{n,{m_j}}}} \right]} } } \right]\left[ {\sum\limits_{n = 0}^{N - 1} {\sum\limits_{{m_0},{m_1}} {\left[ {\prod\limits_{j = 0}^1 {{s_{I,n,{m_j}}}} {A_{n,{m_j}}}} \right]} } } \right]
\end{align}

Also, with the optimum phases ${\mathbf{\Phi }}_I^ \star $, the mutual information $I$ can be rewritten as

\begin{equation}\label{eqn:mutual_information_posynomial}
  I\left( {{{\mathbf{S}}_I},{\mathbf{\Phi }}_I^ \star ,\rho } \right) = {\log _2}\left( {\prod\limits_{n = 0}^{N - 1} {\left( {1 + \frac{{(1 - \rho )}}{{\sigma _n^2}}{C_n}} \right)} } \right)
\end{equation}

with ${C_n} = \sum\nolimits_{{m_0},{m_1}} {\prod\nolimits_{j = 0}^1 {{s_{I,n,{m_j}}}{A_{n,{m_j}}}} } $.

It is also demonstrated in \cite{Clerckx2017} that using a matched filter for power allocation can lead to a suboptimal performance with much lower complexity. Therefore, it is employed to initialize the iterative algorithms for fast convergence. On antenna $m$ and subband $n$, the amplitudes of power and information waveform are initialized to

\begin{equation}\label{eqn:initial_amplitude}
  {s_{P,n,m}} = {s_{I,n,m}} = c{A_{n,m}}
\end{equation}

where $c$ is the coefficient to guarantee the transmit power constraint.

With the optimum phases ${\mathbf{\Phi }}_P^ \star $ and ${\mathbf{\Phi }}_I^ \star $, both the target function \eqref{eqn:target_posynomial} and mutual information \eqref{eqn:mutual_information_posynomial} are posynomials \cite{Boyd2007}. Therefore, we can convert the characterization of R-E region into an energy maximization problem with average transmit power budget $P$ and rate constraint ${\bar R}$

\begin{eqnarray}
  {\mathop {\max }\limits_{{{\mathbf{S}}_P},{{\mathbf{S}}_I},\rho } }&{{z_{DC}}\left( {{{\mathbf{S}}_P},{{\mathbf{S}}_I},{\mathbf{\Phi }}_P^ \star ,{\mathbf{\Phi }}_I^ \star ,\rho } \right)} \label{eqn:original_target}\\
  {{\text{ subject to }}}&{\frac{1}{2}\left[ {\left\| {{{\mathbf{S}}_I}} \right\|_F^2 + \left\| {{{\mathbf{S}}_P}} \right\|_F^2} \right] \leqslant P,} \label{eqn:original_power_constraint} \\
  {}&{I\left( {{{\mathbf{S}}_I},{\mathbf{\Phi }}_I^ \star ,\rho } \right) \geqslant \bar R} \label{eqn:original_rate_constraint}
\end{eqnarray}

In the following section, we transform the optimization into standard Geometric Programming (GP) problems and consider different settings and constraints. 