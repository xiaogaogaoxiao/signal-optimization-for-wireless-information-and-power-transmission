Wireless can be more than communications. As a medium of both information and energy, the radio wave enables a unified Wireless Information and Power Transfer (WIPT) to connect and charge low-power devices remotely. In this paper, we depart from the rectifier behavior and derive a nonlinear energy harvester model to investigate the relationship between waveform design and power delivery. On top of that, a superposition of multi-carrier modulated and unmodulated (multisine) waveforms is introduced to improve the rate-energy (R-E) tradeoff. With an adaptive transceiver design, we jointly optimize the superposed signal at the transmitter and the power splitter at the receiver according to the channel state information. We then extend the existing works to MIMO and consider the influence of PAPR constraints and frequency selectivity. Based on non-convex posynomial maximization, the iterative algorithms are demonstrated to benefit the R-E region especially for multi-carrier transmissions at high SNR. Numerical results also highlight the importance of modelling harvester nonlinearity in WIPT system design. 