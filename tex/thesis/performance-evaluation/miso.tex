Next, we switch to the MISO case and investigate the influence of transmit antenna $M$ on the R-E performance over typical multipath FF and FS channels. Thanks to the decoupling approach, the predetermined beamforming phases ${\mathbf{\Phi }}_I^ \star ,{\mathbf{\Phi }}_P^ \star $ are optimal for MISO and the computational complexity is irrelevant to $M$. Figure \ref{fig:miso-channel} shows the frequency response employed in the optimization for $M = 2$ and 3.

\begin{figure}[ht]
  \centering
  \subfigure[$M = 2$]{
    \includegraphics[width=0.48\textwidth]{miso_frequency_flat_channel_2tx}\label{fig:miso-frequency-flat-channel-2tx}}
  \subfigure[$M = 3$]{
    \includegraphics[width=0.48\textwidth]{miso_frequency_flat_channel_3tx}\label{fig:miso-frequency-flat-channel-3tx}}
  \caption{Frequency response of the MISO FF channels}
  \label{fig:miso-channel}
\end{figure}

The corresponding R-E regions for $N = 4$ and 8 are illustrated in Figure \ref{fig:miso-re}.

\begin{figure}[ht]
  \centering
  \subfigure[$N = 4$]{
    \includegraphics[width=0.48\textwidth]{miso_re_subband_4}\label{fig:miso-re-subband-4}}
  \subfigure[$N = 8$]{
    \includegraphics[width=0.48\textwidth]{miso_re_subband_8}\label{fig:miso-re-subband-8}}
  \caption{R-E region vs $M$ and $N$ over typical FF channels}
  \label{fig:miso-re}
\end{figure}

A first observation is that the concacity-convexity start to appear from $N = 4$ for the MISO systems, in contrast to $N = 8$ in the previous SISO case. It is as expected that the trend becomes more obvious when $N$ increases. Moreover, the plots suggest that a large $M$ can significantly boost the energy benefit of the multisine waveform. In this particular instance, raising $M$ from 2 to 3 produce an current gain of around 400 \%. Although this value partially results from the difference in channel amplitude, it highlights the benefit of using multiple antennas in WIPT. The reason is that increasing $M$ essentially enhances the subchannels such that the terms contributing to the harvested current are amplified. Also, a smaller $N$ is needed to achieve a certain output current level, which further reduces the signal PAPR. Hence, increasing $M$ can be a possible solution for PAPR-constrained devices. Moreover, a combination of TS (between WPT and WIPT) at low rate and PS at high rate guarantees the optimal R-E region as a convex hull.

To eliminate the influence of channel randomness, we investigate the rate and energy performance for 100 FF channels with $N = 4$. Figure \ref{fig:miso-cdf} shows the Cumulative Distribution Function (CDF) of maximum rate and DC current that correspond to WIT and WPT respectively. It is less interesting for WIPT since the variation trend of the R-E tradeoff is not presented. A better representation is required to avoid channel randomness and characterize the general R-E region.

\begin{figure}[ht]
  \centering
  \subfigure[Rate]{
    \includegraphics[width=0.48\textwidth]{miso_cdf_rate}\label{fig:miso-cdf-rate}}
  \subfigure[Current]{
    \includegraphics[width=0.48\textwidth]{miso_cdf_current}\label{fig:miso-cdf-current}}
  \caption{Rate and current CDF vs $M$ for MISO FF channels}
  \label{fig:miso-cdf}
\end{figure} 