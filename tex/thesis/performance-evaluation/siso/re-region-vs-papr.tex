Figure \ref{fig:re-papr} investigates the relationship between PAPR and R-E region for $N = 8, 16$ under the FF and FS channel response as in Figure \ref{fig:siso-channels}.

\begin{figure}[ht]
  \centering
  \subfigure[FF: $N = 8$]{
    \includegraphics[width=0.48\textwidth]{siso_re_ff_papr_8}\label{fig:re-ff-papr-8}}
  \subfigure[FF: $N = 16$]{
    \includegraphics[width=0.48\textwidth]{siso_re_ff_papr_16}\label{fig:re-ff-papr-16}}
  \quad
  \subfigure[FS: $N = 8$]{
    \includegraphics[width=0.48\textwidth]{siso_re_fs_papr_8}\label{fig:re-fs-papr-8}}
  \subfigure[FS: $N = 16$]{
    \includegraphics[width=0.48\textwidth]{siso_re_fs_papr_16}\label{fig:re-fs-papr-16}}
  \caption{R-E region vs PAPR for FF channel}
  \label{fig:re-papr}
\end{figure}

A first observation is that a large enough PAPR is required to fully exploit the power gain of the multisine waveform. For $N = 16$, the R-E region is convex for a PAPR no larger than 20 dB and is concave-convex when it increases to 30 dB. Compare the result with Figure \ref{fig:snr-ff-20db} and \ref{fig:snr-fs-20db}, it can be observed that with a small PAPR constraint of 10 dB, the use of multisine waveform is strictly constrained such that the modulated waveform dominates the transmit signal. Hence, the corresponding R-E plot is similar to the result without power waveform. On the contrary, a PAPR of 30 dB is large enough for the optimal performance of the superposed signal in the low-rate region. It can be concluded that the energy benefit of the multisine waveform indeed comes from the high PAPR. In each cycle, the peak pushes the rectifier output voltage to a high level which decreases slowly in the rest of the period.

A contrast of the R-E plots of $N = 8$ and 16 also suggests a larger $N$ requires higher PAPR to achieve the optimal performance. For instance, the concacity-convexity can be observed for both FF and FS channels with a medium PAPR of 20 dB when $N = 8$. On the other hand, the impact of PAPR for small $N$ is not as significant as for large $N$. This verifies the positive correlation between $N$ and PAPR discussed in section \ref{sec:rectenna-behavior}. It further suggests that although increasing $N$ can effectively boost the harvested energy, the PAPR constraint may limit the use of a very large $N$ in practice. 

It is interesting to notice that the frequency selectivity helps to achieve the optimal R-E region with a smaller PAPR. As shown in Figure \ref{fig:re-fs-papr-8}, a PAPR constraint of 20 dB can guarantee the best tradeoff and a larger budget is unnecessary. However, the transmission over FF channel requires a larger PAPR for optimum behavior. This is because frequency selective channel can further amplify the difference of frequency components such that the received signal is with increased PAPR. 