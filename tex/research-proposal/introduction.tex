A major problem for smart networks as Internet-of-Things (IoT) is the power source. Although batteries have removed the physical boundaries for mobile devices, the limited working time and frequent recharging and replacement have become main restrictions for the development of wireless networks. Recently, with the significant reduction in power requirement of electronics, Simultaneously Wireless Information and Power Transfer (SWIPT) with Radio-Frequency (RF) signals has become a sustainable solution to power the devices while keeping them connected.

The pioneer in \cite{R.Varshney2008} defined a capacity-energy function and investigated the trade-offs for a flat Gaussian channel and typical binary channels. \cite{Grover2010} extended the research to frequency-selective channel. Both works assumed the same received signal is used for Energy Harvesting (EH) and Information Decoding (ID) which is the ideal case. \cite{Zhang2013} proposed two practical receiver designs, namely \textit{Time Switching} (TS) that switches between EH and ID on time basis and \textit{Power Splitting} (PS) that splits the received signal into separate portions. However, the works before \cite{Clerckx2016} are mostly based on an oversimplified linear harvester model. In that paper, the authors derived a tractable nonlinear harvester model for Wireless Power Transfer (WPT) to accurately characterize the rectifier behavior, then performed an adaptive multisine waveform design accordingly. It was demonstrated by circuit simulations that the rectifier nonlinearity brings significant gains to the harvested power and radically influences wireless system design. It was extended to SWIPT in \cite{Clerckx2018} where a superposition of modulated information waveform and multisine power waveform is jointly optimized with the power splitting ratio, according to the Channel State Information (CSI) and rate requirements. The results reported that harvester nonlinearity benefits the rate-energy (R-E) tradeoff and favours a different waveform, modulation and input distribution. Nevertheless, the algorithms rely on iterative Geometric Programming (GP) optimization and the computational complexity increases exponentially with the number of subbands. In comparison, \cite{Park2018} proposed a dual mode SWIPT with an adaptive power management (PM) and ID module that adaptively switches between single-tone and multi-tone transmissions. The former employs a multi-energy level signaling with Phase Shift Keying (PSK) for high rate scenario, while the latter modulates the multisine waveform by Peak-to-Average Power Ratio (PAPR) \cite{Krikidis2019} for power-demanding applications. Nevertheless, the paper only covers the case with Single-Input Single-Output (SISO) and one energy harvester. \cite{Ma2019} designed a generic receiver architecture for Multi-Input Multi-Output (MIMO) WPT with multiple rectifiers and maximizes the harvested power through a joint optimization of beamforming and power splitting ratios.

In this research, we will extend the dual mode SWIPT in \cite{Park2018} to MIMO multi-harvester case and optimize the waveform through a joint optimization of beamforming, power splitting and MS operation based on nonlinear harvester models. First, the transceiver controls the operation mode based on the received power. Then, the signal on each receive antenna is split into information/PAPR and energy components. The former is decoded directly while the latter is reallocated by power splitters before fed into rectifiers. Specifically, when the received power level is relatively low, the splitters will combine all energy branches in one rectifier to enjoy the benefit of harvester nonlinearity. In comparison, when the power is sufficiently high, the components will be equally divided to rectifiers to avoid the low conversion efficiency of diode breakdown region \cite{Clerckx2019}. We will investigate the system performance for Frequency-Flat (FF) and Frequency-Selective (FS) channels.