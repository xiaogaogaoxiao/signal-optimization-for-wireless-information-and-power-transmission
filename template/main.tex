\documentclass[a4paper, twoside]{report}

\usepackage[english]{babel}
\usepackage[utf8x]{inputenc}
\usepackage[T1]{fontenc}
\usepackage{listings}
\usepackage{hyperref}
\hypersetup{colorlinks=false}
\usepackage{lscape}
\usepackage{subfigure}
\usepackage{amsmath}
\usepackage{graphicx}
\usepackage[colorinlistoftodos]{todonotes}

%% Sets page size and margins
\usepackage[a4paper,top=3cm,bottom=2cm,left=3cm,right=3cm,marginparwidth=1.75cm]{geometry}

\title{Project Title}
\author{John Smith}
% Update supervisor and other title stuff in title/title.tex

\begin{document}
\begin{titlepage}

\newcommand{\HRule}{\rule{\linewidth}{0.5mm}} % Defines a new command for the horizontal lines, change thickness here

%----------------------------------------------------------------------------------------
%	LOGO SECTION
%----------------------------------------------------------------------------------------

\includegraphics[width=8cm]{imperial} % Include a department/university logo - this will require the graphicx package

%----------------------------------------------------------------------------------------

\center % Center everything on the page

%----------------------------------------------------------------------------------------
%	HEADING SECTIONS
%----------------------------------------------------------------------------------------
\quad\\[1.5cm]
\textsc{\Large Imperial College London}\\[0.5cm] % Major heading such as course name
\textsc{\large Department of Electrical and Electronic Engineering}\\[1cm] % Minor heading such as course title

%----------------------------------------------------------------------------------------
%	TITLE SECTION
%----------------------------------------------------------------------------------------
\makeatletter
\HRule \\[0.8cm]
{ \LARGE \textbf{\projecttitle}}\\[0.4cm] % Title of your document
\HRule \\[1.5cm]

%----------------------------------------------------------------------------------------
%	AUTHOR SECTION
%----------------------------------------------------------------------------------------

\begin{minipage}[t]{0.31\textwidth}
\centering
\large
\emph{Author:}\\
\articleauthor\\
(CID: \cid) % Your name
\end{minipage}
~
\begin{minipage}[t]{0.31\textwidth}
\centering
\large
\emph{Supervisor:} \\
\projectsupervisor % Supervisor
\end{minipage}
~
\begin{minipage}[t]{0.31\textwidth}
\centering
\large
\emph{Co-Supervisor:} \\
\projectcosupervisor % Co-supervisor
\end{minipage}\\[3cm]
\makeatother


%----------------------------------------------------------------------------------------
%	DATE SECTION
%----------------------------------------------------------------------------------------

{\large A thesis submitted for the degree of}\\[0.5cm]
{\large \emph{MSc Communications and Signal Processing}}\\[0.5cm]
{\large \today}\\[2cm] % Date, change the \today to a set date if you want to be precise

\vfill % Fill the rest of the page with whitespace

\end{titlepage} 

\begin{abstract}
Your abstract goes here. The abstract is a very brief summary of the dissertation's contents. It should be about half a page long. Somebody unfamiliar with your project should have a good idea of what it's about having read the abstract alone and will know whether it will be of interest to them.
\end{abstract}

\renewcommand{\abstractname}{Acknowledgements}
\begin{abstract}
It is usual to thank those individuals who have provided particularly useful assistance, technical or otherwise, during your project.
\end{abstract}

\tableofcontents
\listoffigures
\listoftables

Energy-constrained wireless networks are conventionally powered by batteries. However, limited operation time and high cost in recharging or replacement are restricting the emergence of smart networks as Internet-of-Things (IoT). As a promising solution, Energy Harvesting (EH) from the environment potentially provides perpetual power to the system. Compared with other renewable resources as solar, wind and water, Radio-Frequency (RF) waves typically contain less energy and are more suitable for low-power applications as Wireless Sensor Network (WSN). With the significant reduction in power requirements of chips and processors, Wireless Power Transfer (WPT) that harvests energy from the ambient electromagnetic (EM) waves has received recent attentions in both academia and industry \cite{R.Varshney2008,Grover2010,Zhang2013,Hui2014,Krikidis2014,Valenta2014,Boshkovska2015,Ding2015,Costanzo2016,Clerckx2018a}.

On the other hand, RF radiation has been a medium for Wireless Information Transfer (WIT) for more than a century. Naturally, a unified design of Wireless Information and Power transmission (WIPT) is expected to be a prominent solution to power billions of energy-constrained devices while keeping them connected. \cite{R.Varshney2008} first defined a nonlinear concave capacity-energy function and investigated the tradeoff for some binary channels and a flat AWGN (additive white Gaussian noise) channel with amplitude-constrained inputs. It was extended to frequency-selective channel in \cite{Grover2010}. Both works were based on the ideal assumption that information decoding (ID) and EH can be performed individually on the received signal. In \cite{Zhang2013}, the authors proposed two practical co-located receiver designs named \textit{time switching} (PS) that switches between ID and EH and \textit{power splitting} (TS) that splits the received power into two separate streams. It was demonstrated in \cite{Zhou2013a} that TS can guarantee the same rate as conventional Time-Division Multiple Access (TDMA) while providing reasonable energy. In comparison, PS may lead to higher rate when the power requirement is sufficiently high. A further research \cite{Liu2013} enabled dynamic power splitting that adjusts the power split ratio based on the channel state information (CSI), and proposed a suboptimal low-complexity \textit{antenna switching} scheme. 
Energy-constrained wireless devices are conventionally powered by batteries. However, the development of large-scale networks as Internet-of-Things (IoT) is strictly restricted by its limited working time and frequent recharging or replacement. Although Wireless Power Transfer (WPT) via inductive coupling has enjoyed some success in real-world applications, it is impractical for most devices on the move since the operation range is very short. As a promising alternative, the Radio-Frequency (RF) wave is typically with lower power level (\si{\uW} to \si{W}) but larger coverage (up to hundreds of meters) \cite{Ng2019}. Interestingly, it indeed carries both information and energy simultaneously, with the potential to power billions of mobile nodes wirelessly while keeping them connected. The recent revolution in harvester model and the significant power drop of electronics bring more possibility to the research on Wireless Information and Power Transfer (WIPT) via RF signals.

\chapter{Main sections of the project}
\input{evaluation/evaluation.tex}
\chapter{Conclusions and Future Works}
This paper explored the waveform optimization and rate-energy region characterization for a point-to-point WIPT. Based on the nonlinear rectifier model, a superposition of multi-carrier modulated and unmodulated waveforms at the transmitter are jointly optimized with the power splitter at the receiver. The signal design is modelled as a non-convex posynomial maximization problem adaptive to the CSI. We also extend the existing work to MIMO systems and consider the PAPR constraints.

Numerical results demonstrate the following conclusions. First, the harvester nonlinearity can be exploited to boost the harvested energy. It prefers a different waveform design, transceiver architecture and resource allocation. Second, modulation benefits the delivered power in single-carrier transmission but is detrimental for multi-carrier WIPT. Third, the superposed signal can effectively enlarge the R-E region with the twofold benefit of multisine. Fourth, a combination of PS and TS is generally the optimal receiver strategy. Fifth, frequency selectivity has positive influence on the harvested energy. Sixth, increasing Tx and/or Rx not only improves the rate-energy tradeoff but also reduces the PAPR requirement.

Some limitations of this work require further attention. First, the adaptive design relies on perfect CSIT and synchronization between transmitter and receiver, but both assumptions are hard to implement. Second, the power split ratio can be difficult to adjust dynamically in practice. Third, the GP approach is suboptimal for MIMO systems. Fourth, the iterative algorithms are sensitive to initialization and take long time to solve when a large number of subbands and/or antennas are employed.

Several novel ideas in recent research may be further integrated with this paper. For instance, \cite{Park2018} proposed a dual-mode SWIPT with an adaptive \textit{Mode Switching} (MS) algorithm to alternate between single-tone and multi-tone transmission. The former employs a multi-energy level signaling with Phase Shift Keying (PSK) for high rate communication, while the latter modulates the multisine waveform by PAPR for power-demanding applications \cite{Krikidis2019}. Also, a generic receiver architecture for MIMO WPT was designed in \cite{Ma2019}, which demonstrated that using multiple rectifiers with proper beamforming and power allocation scheme can significantly improve the harvested power. Moreover, another nonlinear EH model was proposed in \cite{Boshkovska2015} whose parameters relies on a logistic curve fitting technique.
\appendix
\chapter{Appendix}
The source code and relevant materials can be retrieved from \href{https://github.com/SnowzTail/signal-optimization-for-wireless-information-and-power-transmission}{[this link]}. 

\bibliographystyle{unsrt}
\bibliography{bibs/sample}

\end{document}