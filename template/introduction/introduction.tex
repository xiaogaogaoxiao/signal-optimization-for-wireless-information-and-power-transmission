\chapter{Introduction}
This is one of the most important components of the dissertation. It should begin with a clear statement of what the project is about so that the nature and scope of the project can be understood by a lay reader. It should summarise everything you set out to achieve, provide a clear summary of the project's background and relevance to other work and give pointers to the remaining sections of the dissertation which contain the bulk of the technical material.

Hello, here's a citation \cite{greenwade93}. This is an inline equation $\Gamma(t)=K_i e^{\sin^2(\omega_t)}$. The first paragraph appears without indent but the following ones will have an indentation.

This is an actual named equation:
\begin{equation}
v(x)=\frac{1}{2}\sin(2 \omega t + \phi) e^{-j s t}
\label{eq:cacona}
\end{equation}
\noindent where $\omega$ is the angular speed. This paragraph didn't have an indentation because the first sentence was linked to the definition of equation (\ref{eq:cacona}). A code snippet for an example program is shown in Listing~\ref{lst:code1}.

\begin{lstlisting}[caption=Source code for {\it hello.m},label=lst:code1,breaklines=true,basewidth=4pt,prebreak=**,postbreak=**,frame=single]
for i:=maxint to 0 do
begin
{ do nothing }
end;
Write('Case insensitive ');
Write('Pascal keywords.');
\end{lstlisting}

The characteristic parameters of the system are sumarised in Table~\ref{tab:tab1}. A figure is shown Fig~\ref{fig:felix}, we don't necessarily know if this figure will appear below, above or elsewhere; therefore, the text should never refer to the figure with sentences such as {\it "As shown here:"}.

\begin{figure}[htbp]
\centering
\includegraphics[width=0.3\linewidth]{introduction/fig/Felix_the_cat.pdf}
\caption{Felix the Cat}
\label{fig:felix}
\end{figure}

\begin{table}[htbp]
	\centering
	\begin{tabular}{lll}
		Parameter & Value & Units\\
		\hline
		$P$ & 1 & kW \\
		$Q$ & 0 & kVAr\\
	    \hline
	\end{tabular}
	\caption{Characteristic parameters of the system}
	\label{tab:tab1}
\end{table}

\begin{landscape}
	\begin{figure}[htbp]
\centering
\includegraphics[width=0.5\linewidth]{introduction/fig/Felix_the_cat.pdf}
\caption{Here's a large drawing of Felix the Cat that wouldn't fit in a portrait page}
\label{fig:felix2}
\end{figure}
\end{landscape}

Other \textbf{things} that might be useful:  Figure \ref{fig:fig2} contains Subfigure \ref{fig:fig2sub1}.

\begin{figure}[htbp]
	\centering
	\subfigure[First one.]{
		\label{fig:fig2sub1}
        \includegraphics[width=0.3\linewidth]{introduction/fig/Felix_the_cat.pdf}}
	\subfigure[Second one.]{
		\label{fig:fig2sub2}
		\includegraphics[width=0.3\linewidth]{introduction/fig/Felix_the_cat.pdf}}
	\caption{A figure with two subfigures.}
	\label{fig:fig2}
\end{figure}

Sometimes, the symbols in an equation are defined as follows\footnote{Some authors like to define their symbols that way.}:
\begin{equation}
	V(t)=A \sin(\omega t+\theta_0)
\end{equation}

\begin{tabular}{lll}
	where & $V$ & is a voltage waveform,\\
	& $A$ & is the amplitude of the voltage,\\
	& $\omega$ & is the angular frequency,\\
	& $t$ & is the time.
\end{tabular}

\section{Objectives}
\section{Challenges}
\section{Contributions}